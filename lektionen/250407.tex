\documentclass[a4paper]{scrreprt}
\usepackage[german]{babel}
\usepackage[utf8]{inputenc}
\usepackage{array}
\usepackage{booktabs}
\usepackage{longtable}
\usepackage{ragged2e}
\usepackage{enumitem}
\setlist[itemize]{noitemsep}

\begin{document}
\section*{2gW, 7. April 25, Einführung Pyhton Anwendungen WR}
\begin{longtable}{p{1.5cm}>{\RaggedRight}p{7.5cm}p{2.5cm}}
    \toprule
    \emph{Zeit}&\emph{Inhalt}&\emph{Methode}\\
    \midrule
    \endhead

    \midrule
    \multicolumn{3}{c}{\begin{tiny}\textit{to be continued}\end{tiny}}\\
    \midrule
    \endfoot

    \bottomrule
    \endlastfoot

    0945&Situierung Vorbereitung Stellvertretung&Vortrag\\ [3pt]
        &Problem: Kündigung als Empfangsbedürftige Erklärung&\\
        &Repetition Kündigungsfristen im Mietrecht&Lehrgespräch\\ [3pt]
    0950&Aufsetzen der Arbeitsumgebung&Einzelarbeit\\
        &\vspace{-7.5mm}
         \begin{itemize}
             \item[--] Übernahme des Arbeitsblattes nach Colab oder
             \item[--] Aufsetzen eines Ordners für WR-Anwendungen 
         \end{itemize}&\\ [3pt]

    0955&Einführung Python \texttt{datetime} Library&Vortrag\\
        &Ausprobieren mit dem eigenen Geburtstag&\\ [3pt]

    1000&Einführung Python \texttt{holiday} Library&Vortrag\\
        &Ausprobieren mit dem 1. Mai&\\

    1005&Funktion für die Berechnung des Zustelldatums&Lehrgespräch\\
        &Zustelldauer, Berücksichtigung von Wochenenden und
        Feiertagen&\\ [3pt]

    1010&Implementation der Funktion
    \texttt{get\_zustelldatum}&Einzelarbeit\\ [3pt]

    1025&Besprechung des Resultates&Lehrgespräch\\


\end{longtable}
\end{document}
